\batchmode %% Suppresses most terminal output.
\documentclass{article}
\usepackage{color}
\definecolor{boxshade}{gray}{0.85}
\setlength{\textwidth}{360pt}
\setlength{\textheight}{541pt}
\usepackage{latexsym}
\usepackage{ifthen}
% \usepackage{color}
%%%%%%%%%%%%%%%%%%%%%%%%%%%%%%%%%%%%%%%%%%%%%%%%%%%%%%%%%%%%%%%%%%%%%%%%%%%%%
% SWITCHES                                                                  %
%%%%%%%%%%%%%%%%%%%%%%%%%%%%%%%%%%%%%%%%%%%%%%%%%%%%%%%%%%%%%%%%%%%%%%%%%%%%%
\newboolean{shading} 
\setboolean{shading}{false}
\makeatletter
 %% this is needed only when inserted into the file, not when
 %% used as a package file.
%%%%%%%%%%%%%%%%%%%%%%%%%%%%%%%%%%%%%%%%%%%%%%%%%%%%%%%%%%%%%%%%%%%%%%%%%%%%%
%                                                                           %
% DEFINITIONS OF SYMBOL-PRODUCING COMMANDS                                  %
%                                                                           %
%    TLA+      LaTeX                                                        %
%    symbol    command                                                      %
%    ------    -------                                                      %
%    =>        \implies                                                     %
%    <:        \ltcolon                                                     %
%    :>        \colongt                                                     %
%    ==        \defeq                                                       %
%    ..        \dotdot                                                      %
%    ::        \coloncolon                                                  %
%    =|        \eqdash                                                      %
%    ++        \pp                                                          %
%    --        \mm                                                          %
%    **        \stst                                                        %
%    //        \slsl                                                        %
%    ^         \ct                                                          %
%    \A        \A                                                           %
%    \E        \E                                                           %
%    \AA       \AA                                                          %
%    \EE       \EE                                                          %
%%%%%%%%%%%%%%%%%%%%%%%%%%%%%%%%%%%%%%%%%%%%%%%%%%%%%%%%%%%%%%%%%%%%%%%%%%%%%
\newlength{\symlength}
\newcommand{\implies}{\Rightarrow}
\newcommand{\ltcolon}{\mathrel{<\!\!\mbox{:}}}
\newcommand{\colongt}{\mathrel{\!\mbox{:}\!\!>}}
\newcommand{\defeq}{\;\mathrel{\smash   %% keep this symbol from being too tall
    {{\stackrel{\scriptscriptstyle\Delta}{=}}}}\;}
\newcommand{\dotdot}{\mathrel{\ldotp\ldotp}}
\newcommand{\coloncolon}{\mathrel{::\;}}
\newcommand{\eqdash}{\mathrel = \joinrel \hspace{-.28em}|}
\newcommand{\pp}{\mathbin{++}}
\newcommand{\mm}{\mathbin{--}}
\newcommand{\stst}{*\!*}
\newcommand{\slsl}{/\!/}
\newcommand{\ct}{\hat{\hspace{.4em}}}
\newcommand{\A}{\forall}
\newcommand{\E}{\exists}
\renewcommand{\AA}{\makebox{$\raisebox{.05em}{\makebox[0pt][l]{%
   $\forall\hspace{-.517em}\forall\hspace{-.517em}\forall$}}%
   \forall\hspace{-.517em}\forall \hspace{-.517em}\forall\,$}}
\newcommand{\EE}{\makebox{$\raisebox{.05em}{\makebox[0pt][l]{%
   $\exists\hspace{-.517em}\exists\hspace{-.517em}\exists$}}%
   \exists\hspace{-.517em}\exists\hspace{-.517em}\exists\,$}}
\newcommand{\whileop}{\.{\stackrel
  {\mbox{\raisebox{-.3em}[0pt][0pt]{$\scriptscriptstyle+\;\,$}}}%
  {-\hspace{-.16em}\triangleright}}}

% Commands are defined to produce the upper-case keywords.
% Note that some have space after them.
\newcommand{\ASSUME}{\textsc{assume }}
\newcommand{\ASSUMPTION}{\textsc{assumption }}
\newcommand{\AXIOM}{\textsc{axiom }}
\newcommand{\BOOLEAN}{\textsc{boolean }}
\newcommand{\CASE}{\textsc{case }}
\newcommand{\CONSTANT}{\textsc{constant }}
\newcommand{\CONSTANTS}{\textsc{constants }}
\newcommand{\ELSE}{\settowidth{\symlength}{\THEN}%
   \makebox[\symlength][l]{\textsc{ else}}}
\newcommand{\EXCEPT}{\textsc{ except }}
\newcommand{\EXTENDS}{\textsc{extends }}
\newcommand{\FALSE}{\textsc{false}}
\newcommand{\IF}{\textsc{if }}
\newcommand{\IN}{\settowidth{\symlength}{\LET}%
   \makebox[\symlength][l]{\textsc{in}}}
\newcommand{\INSTANCE}{\textsc{instance }}
\newcommand{\LET}{\textsc{let }}
\newcommand{\LOCAL}{\textsc{local }}
\newcommand{\MODULE}{\textsc{module }}
\newcommand{\OTHER}{\textsc{other }}
\newcommand{\STRING}{\textsc{string}}
\newcommand{\THEN}{\textsc{ then }}
\newcommand{\THEOREM}{\textsc{theorem }}
\newcommand{\LEMMA}{\textsc{lemma }}
\newcommand{\PROPOSITION}{\textsc{proposition }}
\newcommand{\COROLLARY}{\textsc{corollary }}
\newcommand{\TRUE}{\textsc{true}}
\newcommand{\VARIABLE}{\textsc{variable }}
\newcommand{\VARIABLES}{\textsc{variables }}
\newcommand{\WITH}{\textsc{ with }}
\newcommand{\WF}{\textrm{WF}}
\newcommand{\SF}{\textrm{SF}}
\newcommand{\CHOOSE}{\textsc{choose }}
\newcommand{\ENABLED}{\textsc{enabled }}
\newcommand{\UNCHANGED}{\textsc{unchanged }}
\newcommand{\SUBSET}{\textsc{subset }}
\newcommand{\UNION}{\textsc{union }}
\newcommand{\DOMAIN}{\textsc{domain }}
% Added for tla2tex
\newcommand{\BY}{\textsc{by }}
\newcommand{\OBVIOUS}{\textsc{obvious }}
\newcommand{\HAVE}{\textsc{have }}
\newcommand{\QED}{\textsc{qed }}
\newcommand{\TAKE}{\textsc{take }}
\newcommand{\DEF}{\textsc{ def }}
\newcommand{\HIDE}{\textsc{hide }}
\newcommand{\RECURSIVE}{\textsc{recursive }}
\newcommand{\USE}{\textsc{use }}
\newcommand{\DEFINE}{\textsc{define }}
\newcommand{\PROOF}{\textsc{proof }}
\newcommand{\WITNESS}{\textsc{witness }}
\newcommand{\PICK}{\textsc{pick }}
\newcommand{\DEFS}{\textsc{defs }}
\newcommand{\PROVE}{\settowidth{\symlength}{\ASSUME}%
   \makebox[\symlength][l]{\textsc{prove}}\@s{-4.1}}%
  %% The \@s{-4.1) is a kludge added on 24 Oct 2009 [happy birthday, Ellen]
  %% so the correct alignment occurs if the user types
  %%   ASSUME X
  %%   PROVE  Y
  %% because it cancels the extra 4.1 pts added because of the 
  %% extra space after the PROVE.  This seems to works OK.
  %% However, the 4.1 equals Parameters.LaTeXLeftSpace(1) and
  %% should be changed if that method ever changes.
\newcommand{\SUFFICES}{\textsc{suffices }}
\newcommand{\NEW}{\textsc{new }}
\newcommand{\LAMBDA}{\textsc{lambda }}
\newcommand{\STATE}{\textsc{state }}
\newcommand{\ACTION}{\textsc{action }}
\newcommand{\TEMPORAL}{\textsc{temporal }}
\newcommand{\ONLY}{\textsc{only }}              %% added by LL on 2 Oct 2009
\newcommand{\OMITTED}{\textsc{omitted }}        %% added by LL on 31 Oct 2009
\newcommand{\@pfstepnum}[2]{\ensuremath{\langle#1\rangle}\textrm{#2}}
\newcommand{\bang}{\@s{1}\mbox{\small !}\@s{1}}
%% We should format || differently in PlusCal code than in TLA+ formulas.
\newcommand{\p@barbar}{\ifpcalsymbols
   \,\,\rule[-.25em]{.075em}{1em}\hspace*{.2em}\rule[-.25em]{.075em}{1em}\,\,%
   \else \,||\,\fi}
%% PlusCal keywords
\newcommand{\p@fair}{\textbf{fair }}
\newcommand{\p@semicolon}{\textbf{\,; }}
\newcommand{\p@algorithm}{\textbf{algorithm }}
\newcommand{\p@mmfair}{\textbf{-{}-fair }}
\newcommand{\p@mmalgorithm}{\textbf{-{}-algorithm }}
\newcommand{\p@assert}{\textbf{assert }}
\newcommand{\p@await}{\textbf{await }}
\newcommand{\p@begin}{\textbf{begin }}
\newcommand{\p@end}{\textbf{end }}
\newcommand{\p@call}{\textbf{call }}
\newcommand{\p@define}{\textbf{define }}
\newcommand{\p@do}{\textbf{ do }}
\newcommand{\p@either}{\textbf{either }}
\newcommand{\p@or}{\textbf{or }}
\newcommand{\p@goto}{\textbf{goto }}
\newcommand{\p@if}{\textbf{if }}
\newcommand{\p@then}{\,\,\textbf{then }}
\newcommand{\p@else}{\ifcsyntax \textbf{else } \else \,\,\textbf{else }\fi}
\newcommand{\p@elsif}{\,\,\textbf{elsif }}
\newcommand{\p@macro}{\textbf{macro }}
\newcommand{\p@print}{\textbf{print }}
\newcommand{\p@procedure}{\textbf{procedure }}
\newcommand{\p@process}{\textbf{process }}
\newcommand{\p@return}{\textbf{return}}
\newcommand{\p@skip}{\textbf{skip}}
\newcommand{\p@variable}{\textbf{variable }}
\newcommand{\p@variables}{\textbf{variables }}
\newcommand{\p@while}{\textbf{while }}
\newcommand{\p@when}{\textbf{when }}
\newcommand{\p@with}{\textbf{with }}
\newcommand{\p@lparen}{\textbf{(\,\,}}
\newcommand{\p@rparen}{\textbf{\,\,) }}   
\newcommand{\p@lbrace}{\textbf{\{\,\,}}   
\newcommand{\p@rbrace}{\textbf{\,\,\} }}

%%%%%%%%%%%%%%%%%%%%%%%%%%%%%%%%%%%%%%%%%%%%%%%%%%%%%%%%%
% REDEFINE STANDARD COMMANDS TO MAKE THEM FORMAT BETTER %
%                                                       %
% We redefine \in and \notin                            %
%%%%%%%%%%%%%%%%%%%%%%%%%%%%%%%%%%%%%%%%%%%%%%%%%%%%%%%%%
\renewcommand{\_}{\rule{.4em}{.06em}\hspace{.05em}}
\newlength{\equalswidth}
\let\oldin=\in
\let\oldnotin=\notin
\renewcommand{\in}{%
   {\settowidth{\equalswidth}{$\.{=}$}\makebox[\equalswidth][c]{$\oldin$}}}
\renewcommand{\notin}{%
   {\settowidth{\equalswidth}{$\.{=}$}\makebox[\equalswidth]{$\oldnotin$}}}


%%%%%%%%%%%%%%%%%%%%%%%%%%%%%%%%%%%%%%%%%%%%%%%%%%%%
%                                                  %
% HORIZONTAL BARS:                                 %
%                                                  %
%   \moduleLeftDash    |~~~~~~~~~~                 %
%   \moduleRightDash    ~~~~~~~~~~|                %
%   \midbar            |----------|                %
%   \bottombar         |__________|                %
%%%%%%%%%%%%%%%%%%%%%%%%%%%%%%%%%%%%%%%%%%%%%%%%%%%%
\newlength{\charwidth}\settowidth{\charwidth}{{\small\tt M}}
\newlength{\boxrulewd}\setlength{\boxrulewd}{.4pt}
\newlength{\boxlineht}\setlength{\boxlineht}{.5\baselineskip}
\newcommand{\boxsep}{\charwidth}
\newlength{\boxruleht}\setlength{\boxruleht}{.5ex}
\newlength{\boxruledp}\setlength{\boxruledp}{-\boxruleht}
\addtolength{\boxruledp}{\boxrulewd}
\newcommand{\boxrule}{\leaders\hrule height \boxruleht depth \boxruledp
                      \hfill\mbox{}}
\newcommand{\@computerule}{%
  \setlength{\boxruleht}{.5ex}%
  \setlength{\boxruledp}{-\boxruleht}%
  \addtolength{\boxruledp}{\boxrulewd}}

\newcommand{\bottombar}{\hspace{-\boxsep}%
  \raisebox{-\boxrulewd}[0pt][0pt]{\rule[.5ex]{\boxrulewd}{\boxlineht}}%
  \boxrule
  \raisebox{-\boxrulewd}[0pt][0pt]{%
      \rule[.5ex]{\boxrulewd}{\boxlineht}}\hspace{-\boxsep}\vspace{0pt}}

\newcommand{\moduleLeftDash}%
   {\hspace*{-\boxsep}%
     \raisebox{-\boxlineht}[0pt][0pt]{\rule[.5ex]{\boxrulewd
               }{\boxlineht}}%
    \boxrule\hspace*{.4em }}

\newcommand{\moduleRightDash}%
    {\hspace*{.4em}\boxrule
    \raisebox{-\boxlineht}[0pt][0pt]{\rule[.5ex]{\boxrulewd
               }{\boxlineht}}\hspace{-\boxsep}}%\vspace{.2em}

\newcommand{\midbar}{\hspace{-\boxsep}\raisebox{-.5\boxlineht}[0pt][0pt]{%
   \rule[.5ex]{\boxrulewd}{\boxlineht}}\boxrule\raisebox{-.5\boxlineht%
   }[0pt][0pt]{\rule[.5ex]{\boxrulewd}{\boxlineht}}\hspace{-\boxsep}}

%%%%%%%%%%%%%%%%%%%%%%%%%%%%%%%%%%%%%%%%%%%%%%%%%%%%%%%%%%%%%%%%%%%%%%%%%%%%%
% FORMATING COMMANDS                                                        %
%%%%%%%%%%%%%%%%%%%%%%%%%%%%%%%%%%%%%%%%%%%%%%%%%%%%%%%%%%%%%%%%%%%%%%%%%%%%%

%%%%%%%%%%%%%%%%%%%%%%%%%%%%%%%%%%%%%%%%%%%%%%%%%%%%%%%%%%%%%%%%%%%%%%%%%%%%%
% PLUSCAL SHADING                                                           %
%%%%%%%%%%%%%%%%%%%%%%%%%%%%%%%%%%%%%%%%%%%%%%%%%%%%%%%%%%%%%%%%%%%%%%%%%%%%%

% The TeX pcalshading switch is set on to cause PlusCal shading to be
% performed.  This changes the behavior of the following commands and
% environments to cause full-width shading to be performed on all lines.
% 
%   \tstrut \@x cpar mcom \@pvspace
% 
% The TeX pcalsymbols switch is turned on when typesetting a PlusCal algorithm,
% whether or not shading is being performed.  It causes symbols (other than
% parentheses and braces and PlusCal-only keywords) that should be typeset
% differently depending on whether they are in an algorithm to be typeset
% appropriately.  Currently, the only such symbol is "||".
%
% The TeX csyntax switch is turned on when typesetting a PlusCal algorithm in
% c-syntax.  This allows symbols to be format differently in the two syntaxes.
% The "else" keyword is the only one that is.

\newif\ifpcalshading \pcalshadingfalse
\newif\ifpcalsymbols \pcalsymbolsfalse
\newif\ifcsyntax     \csyntaxtrue

% The \@pvspace command makes a vertical space.  It uses \vspace
% except with \ifpcalshading, in which case it sets \pvcalvspace
% and the space is added by a following \@x command.
%
\newlength{\pcalvspace}\setlength{\pcalvspace}{0pt}%
\newcommand{\@pvspace}[1]{%
  \ifpcalshading
     \par\global\setlength{\pcalvspace}{#1}%
  \else
     \par\vspace{#1}%
  \fi
}

% The lcom environment was changed to set \lcomindent equal to
% the indentation it produces.  This length is used by the
% cpar environment to make shading extend for the full width
% of the line.  This assumes that lcom environments are not
% nested.  I hope TLATeX does not nest them.
%
\newlength{\lcomindent}%
\setlength{\lcomindent}{0pt}%

%\tstrut: A strut to produce inter-paragraph space in a comment.
%\rstrut: A strut to extend the bottom of a one-line comment so
%         there's no break in the shading between comments on 
%         successive lines.
\newcommand\tstrut%
  {\raisebox{\vshadelen}{\raisebox{-.25em}{\rule{0pt}{1.15em}}}%
   \global\setlength{\vshadelen}{0pt}}
\newcommand\rstrut{\raisebox{-.25em}{\rule{0pt}{1.15em}}%
 \global\setlength{\vshadelen}{0pt}}


% \.{op} formats operator op in math mode with empty boxes on either side.
% Used because TeX otherwise vary the amount of space it leaves around op.
\renewcommand{\.}[1]{\ensuremath{\mbox{}#1\mbox{}}}

% \@s{n} produces an n-point space
\newcommand{\@s}[1]{\hspace{#1pt}}           

% \@x{txt} starts a specification line in the beginning with txt
% in the final LaTeX source.
\newlength{\@xlen}
\newcommand\xtstrut%
  {\setlength{\@xlen}{1.05em}%
   \addtolength{\@xlen}{\pcalvspace}%
    \raisebox{\vshadelen}{\raisebox{-.25em}{\rule{0pt}{\@xlen}}}%
   \global\setlength{\vshadelen}{0pt}%
   \global\setlength{\pcalvspace}{0pt}}

\newcommand{\@x}[1]{\par
  \ifpcalshading
  \makebox[0pt][l]{\shadebox{\xtstrut\hspace*{\textwidth}}}%
  \fi
  \mbox{$\mbox{}#1\mbox{}$}}  

% \@xx{txt} continues a specification line with the text txt.
\newcommand{\@xx}[1]{\mbox{$\mbox{}#1\mbox{}$}}  

% \@y{cmt} produces a one-line comment.
\newcommand{\@y}[1]{\mbox{\footnotesize\hspace{.65em}%
  \ifthenelse{\boolean{shading}}{%
      \shadebox{#1\hspace{-\the\lastskip}\rstrut}}%
               {#1\hspace{-\the\lastskip}\rstrut}}}

% \@z{cmt} produces a zero-width one-line comment.
\newcommand{\@z}[1]{\makebox[0pt][l]{\footnotesize
  \ifthenelse{\boolean{shading}}{%
      \shadebox{#1\hspace{-\the\lastskip}\rstrut}}%
               {#1\hspace{-\the\lastskip}\rstrut}}}


% \@w{str} produces the TLA+ string "str".
\newcommand{\@w}[1]{\textsf{``{#1}''}}             


%%%%%%%%%%%%%%%%%%%%%%%%%%%%%%%%%%%%%%%%%%%%%%%%%%%%%%%%%%%%%%%%%%%%%%%%%%%%%
% SHADING                                                                   %
%%%%%%%%%%%%%%%%%%%%%%%%%%%%%%%%%%%%%%%%%%%%%%%%%%%%%%%%%%%%%%%%%%%%%%%%%%%%%
\def\graymargin{1}
  % The number of points of margin in the shaded box.

% \definecolor{boxshade}{gray}{.85}
% Defines the darkness of the shading: 1 = white, 0 = black
% Added by TLATeX only if needed.

% \shadebox{txt} puts txt in a shaded box.
\newlength{\templena}
\newlength{\templenb}
\newsavebox{\tempboxa}
\newcommand{\shadebox}[1]{{\setlength{\fboxsep}{\graymargin pt}%
     \savebox{\tempboxa}{#1}%
     \settoheight{\templena}{\usebox{\tempboxa}}%
     \settodepth{\templenb}{\usebox{\tempboxa}}%
     \hspace*{-\fboxsep}\raisebox{0pt}[\templena][\templenb]%
        {\colorbox{boxshade}{\usebox{\tempboxa}}}\hspace*{-\fboxsep}}}

% \vshade{n} makes an n-point inter-paragraph space, with
%  shading if the `shading' flag is true.
\newlength{\vshadelen}
\setlength{\vshadelen}{0pt}
\newcommand{\vshade}[1]{\ifthenelse{\boolean{shading}}%
   {\global\setlength{\vshadelen}{#1pt}}%
   {\vspace{#1pt}}}

\newlength{\boxwidth}
\newlength{\multicommentdepth}

%%%%%%%%%%%%%%%%%%%%%%%%%%%%%%%%%%%%%%%%%%%%%%%%%%%%%%%%%%%%%%%%%%%%%%%%%%%%%
% THE cpar ENVIRONMENT                                                      %
% ^^^^^^^^^^^^^^^^^^^^                                                      %
% The LaTeX input                                                           %
%                                                                           %
%   \begin{cpar}{pop}{nest}{isLabel}{d}{e}{arg6}                            %
%     XXXXXXXXXXXXXXX                                                       %
%     XXXXXXXXXXXXXXX                                                       %
%     XXXXXXXXXXXXXXX                                                       %
%   \end{cpar}                                                              %
%                                                                           %
% produces one of two possible results.  If isLabel is the letter "T",      %
% it produces the following, where [label] is the result of typesetting     %
% arg6 in an LR box, and d is is a number representing a distance in        %
% points.                                                                   %
%                                                                           %
%   prevailing |<-- d -->[label]<- e ->XXXXXXXXXXXXXXX                      %
%         left |                       XXXXXXXXXXXXXXX                      %
%       margin |                       XXXXXXXXXXXXXXX                      %
%                                                                           %
% If isLabel is the letter "F", then it produces                            %
%                                                                           %
%   prevailing |<-- d -->XXXXXXXXXXXXXXXXXXXXXXX                            %
%         left |         <- e ->XXXXXXXXXXXXXXXX                            %
%       margin |                XXXXXXXXXXXXXXXX                            %
%                                                                           %
% where d and e are numbers representing distances in points.               %
%                                                                           %
% The prevailing left margin is the one in effect before the most recent    %
% pop (argument 1) cpar environments with "T" as the nest argument, where   %
% pop is a number \geq 0.                                                   %
%                                                                           %
% If the nest argument is the letter "T", then the prevailing left          %
% margin is moved to the left of the second (and following) lines of        %
% X's.  Otherwise, the prevailing left margin is left unchanged.            %
%                                                                           %
% An \unnest{n} command moves the prevailing left margin to where it was    %
% before the most recent n cpar environments with "T" as the nesting        %
% argument.                                                                 %
%                                                                           %
% The environment leaves no vertical space above or below it, or between    %
% its paragraphs.  (TLATeX inserts the proper amount of vertical space.)    %
%%%%%%%%%%%%%%%%%%%%%%%%%%%%%%%%%%%%%%%%%%%%%%%%%%%%%%%%%%%%%%%%%%%%%%%%%%%%%

\newcounter{pardepth}
\setcounter{pardepth}{0}

% \setgmargin{txt} defines \gmarginN to be txt, where N is \roman{pardepth}.
% \thegmargin equals \gmarginN, where N is \roman{pardepth}.
\newcommand{\setgmargin}[1]{%
  \expandafter\xdef\csname gmargin\roman{pardepth}\endcsname{#1}}
\newcommand{\thegmargin}{\csname gmargin\roman{pardepth}\endcsname}
\newcommand{\gmargin}{0pt}

\newsavebox{\tempsbox}

\newlength{\@cparht}
\newlength{\@cpardp}
\newenvironment{cpar}[6]{%
  \addtocounter{pardepth}{-#1}%
  \ifthenelse{\boolean{shading}}{\par\begin{lrbox}{\tempsbox}%
                                 \begin{minipage}[t]{\linewidth}}{}%
  \begin{list}{}{%
     \edef\temp{\thegmargin}
     \ifthenelse{\equal{#3}{T}}%
       {\settowidth{\leftmargin}{\hspace{\temp}\footnotesize #6\hspace{#5pt}}%
        \addtolength{\leftmargin}{#4pt}}%
       {\setlength{\leftmargin}{#4pt}%
        \addtolength{\leftmargin}{#5pt}%
        \addtolength{\leftmargin}{\temp}%
        \setlength{\itemindent}{-#5pt}}%
      \ifthenelse{\equal{#2}{T}}{\addtocounter{pardepth}{1}%
                                 \setgmargin{\the\leftmargin}}{}%
      \setlength{\labelwidth}{0pt}%
      \setlength{\labelsep}{0pt}%
      \setlength{\itemindent}{-\leftmargin}%
      \setlength{\topsep}{0pt}%
      \setlength{\parsep}{0pt}%
      \setlength{\partopsep}{0pt}%
      \setlength{\parskip}{0pt}%
      \setlength{\itemsep}{0pt}
      \setlength{\itemindent}{#4pt}%
      \addtolength{\itemindent}{-\leftmargin}}%
   \ifthenelse{\equal{#3}{T}}%
      {\item[\tstrut\footnotesize \hspace{\temp}{#6}\hspace{#5pt}]
        }%
      {\item[\tstrut\hspace{\temp}]%
         }%
   \footnotesize}
 {\hspace{-\the\lastskip}\tstrut
 \end{list}%
  \ifthenelse{\boolean{shading}}%
          {\end{minipage}%
           \end{lrbox}%
           \ifpcalshading
             \setlength{\@cparht}{\ht\tempsbox}%
             \setlength{\@cpardp}{\dp\tempsbox}%
             \addtolength{\@cparht}{.15em}%
             \addtolength{\@cpardp}{.2em}%
             \addtolength{\@cparht}{\@cpardp}%
            % I don't know what's going on here.  I want to add a
            % \pcalvspace high shaded line, but I don't know how to
            % do it.  A little trial and error shows that the following
            % does a reasonable job approximating that, eliminating
            % the line if \pcalvspace is small.
            \addtolength{\@cparht}{\pcalvspace}%
             \ifdim \pcalvspace > .8em
               \addtolength{\pcalvspace}{-.2em}%
               \hspace*{-\lcomindent}%
               \shadebox{\rule{0pt}{\pcalvspace}\hspace*{\textwidth}}\par
               \global\setlength{\pcalvspace}{0pt}%
               \fi
             \hspace*{-\lcomindent}%
             \makebox[0pt][l]{\raisebox{-\@cpardp}[0pt][0pt]{%
                 \shadebox{\rule{0pt}{\@cparht}\hspace*{\textwidth}}}}%
             \hspace*{\lcomindent}\usebox{\tempsbox}%
             \par
           \else
             \shadebox{\usebox{\tempsbox}}\par
           \fi}%
           {}%
  }

%%%%%%%%%%%%%%%%%%%%%%%%%%%%%%%%%%%%%%%%%%%%%%%%%%%%%%%%%%%%%%%%%%%%%%%%%%%%%%
% THE ppar ENVIRONMENT                                                       %
% ^^^^^^^^^^^^^^^^^^^^                                                       %
% The environment                                                            %
%                                                                            %
%   \begin{ppar} ... \end{ppar}                                              %
%                                                                            %
% is equivalent to                                                           %
%                                                                            %
%   \begin{cpar}{0}{F}{F}{0}{0}{} ... \end{cpar}                             %
%                                                                            %
% The environment is put around each line of the output for a PlusCal        %
% algorithm.                                                                 %
%%%%%%%%%%%%%%%%%%%%%%%%%%%%%%%%%%%%%%%%%%%%%%%%%%%%%%%%%%%%%%%%%%%%%%%%%%%%%%
%\newenvironment{ppar}{%
%  \ifthenelse{\boolean{shading}}{\par\begin{lrbox}{\tempsbox}%
%                                 \begin{minipage}[t]{\linewidth}}{}%
%  \begin{list}{}{%
%     \edef\temp{\thegmargin}
%        \setlength{\leftmargin}{0pt}%
%        \addtolength{\leftmargin}{\temp}%
%        \setlength{\itemindent}{0pt}%
%      \setlength{\labelwidth}{0pt}%
%      \setlength{\labelsep}{0pt}%
%      \setlength{\itemindent}{-\leftmargin}%
%      \setlength{\topsep}{0pt}%
%      \setlength{\parsep}{0pt}%
%      \setlength{\partopsep}{0pt}%
%      \setlength{\parskip}{0pt}%
%      \setlength{\itemsep}{0pt}
%      \setlength{\itemindent}{0pt}%
%      \addtolength{\itemindent}{-\leftmargin}}%
%      \item[\tstrut\hspace{\temp}]}%
% {\hspace{-\the\lastskip}\tstrut
% \end{list}%
%  \ifthenelse{\boolean{shading}}{\end{minipage}  
%                                 \end{lrbox}%
%                                 \shadebox{\usebox{\tempsbox}}\par}{}%
%  }

 %%% TESTING
 \newcommand{\xtest}[1]{\par
 \makebox[0pt][l]{\shadebox{\xtstrut\hspace*{\textwidth}}}%
 \mbox{$\mbox{}#1\mbox{}$}} 

% \newcommand{\xxtest}[1]{\par
% \makebox[0pt][l]{\shadebox{\xtstrut{#1}\hspace*{\textwidth}}}%
% \mbox{$\mbox{}#1\mbox{}$}} 

%\newlength{\pcalvspace}
%\setlength{\pcalvspace}{0pt}
% \newlength{\xxtestlen}
% \setlength{\xxtestlen}{0pt}
% \newcommand\xtstrut%
%   {\setlength{\xxtestlen}{1.15em}%
%    \addtolength{\xxtestlen}{\pcalvspace}%
%     \raisebox{\vshadelen}{\raisebox{-.25em}{\rule{0pt}{\xxtestlen}}}%
%    \global\setlength{\vshadelen}{0pt}%
%    \global\setlength{\pcalvspace}{0pt}}
   
   %%%% TESTING
   
   %% The xcpar environment
   %%  Note: overloaded use of \pcalvspace for testing.
   %%
%   \newlength{\xcparht}%
%   \newlength{\xcpardp}%
   
%   \newenvironment{xcpar}[6]{%
%  \addtocounter{pardepth}{-#1}%
%  \ifthenelse{\boolean{shading}}{\par\begin{lrbox}{\tempsbox}%
%                                 \begin{minipage}[t]{\linewidth}}{}%
%  \begin{list}{}{%
%     \edef\temp{\thegmargin}%
%     \ifthenelse{\equal{#3}{T}}%
%       {\settowidth{\leftmargin}{\hspace{\temp}\footnotesize #6\hspace{#5pt}}%
%        \addtolength{\leftmargin}{#4pt}}%
%       {\setlength{\leftmargin}{#4pt}%
%        \addtolength{\leftmargin}{#5pt}%
%        \addtolength{\leftmargin}{\temp}%
%        \setlength{\itemindent}{-#5pt}}%
%      \ifthenelse{\equal{#2}{T}}{\addtocounter{pardepth}{1}%
%                                 \setgmargin{\the\leftmargin}}{}%
%      \setlength{\labelwidth}{0pt}%
%      \setlength{\labelsep}{0pt}%
%      \setlength{\itemindent}{-\leftmargin}%
%      \setlength{\topsep}{0pt}%
%      \setlength{\parsep}{0pt}%
%      \setlength{\partopsep}{0pt}%
%      \setlength{\parskip}{0pt}%
%      \setlength{\itemsep}{0pt}%
%      \setlength{\itemindent}{#4pt}%
%      \addtolength{\itemindent}{-\leftmargin}}%
%   \ifthenelse{\equal{#3}{T}}%
%      {\item[\xtstrut\footnotesize \hspace{\temp}{#6}\hspace{#5pt}]%
%        }%
%      {\item[\xtstrut\hspace{\temp}]%
%         }%
%   \footnotesize}
% {\hspace{-\the\lastskip}\tstrut
% \end{list}%
%  \ifthenelse{\boolean{shading}}{\end{minipage}  
%                                 \end{lrbox}%
%                                 \setlength{\xcparht}{\ht\tempsbox}%
%                                 \setlength{\xcpardp}{\dp\tempsbox}%
%                                 \addtolength{\xcparht}{.15em}%
%                                 \addtolength{\xcpardp}{.2em}%
%                                 \addtolength{\xcparht}{\xcpardp}%
%                                 \hspace*{-\lcomindent}%
%                                 \makebox[0pt][l]{\raisebox{-\xcpardp}[0pt][0pt]{%
%                                      \shadebox{\rule{0pt}{\xcparht}\hspace*{\textwidth}}}}%
%                                 \hspace*{\lcomindent}\usebox{\tempsbox}%
%                                 \par}{}%
%  }
%  
% \newlength{\xmcomlen}
%\newenvironment{xmcom}[1]{%
%  \setcounter{pardepth}{0}%
%  \hspace{.65em}%
%  \begin{lrbox}{\alignbox}\sloppypar%
%      \setboolean{shading}{false}%
%      \setlength{\boxwidth}{#1pt}%
%      \addtolength{\boxwidth}{-.65em}%
%      \begin{minipage}[t]{\boxwidth}\footnotesize
%      \parskip=0pt\relax}%
%       {\end{minipage}\end{lrbox}%
%       \setlength{\xmcomlen}{\textwidth}%
%       \addtolength{\xmcomlen}{-\wd\alignbox}%
%       \settodepth{\alignwidth}{\usebox{\alignbox}}%
%       \global\setlength{\multicommentdepth}{\alignwidth}%
%       \setlength{\boxwidth}{\alignwidth}%
%       \global\addtolength{\alignwidth}{-\maxdepth}%
%       \addtolength{\boxwidth}{.1em}%
%       \raisebox{0pt}[0pt][0pt]{%
%        \ifthenelse{\boolean{shading}}%
%          {\hspace*{-\xmcomlen}\shadebox{\rule[-\boxwidth]{0pt}{0pt}%
%                                 \hspace*{\xmcomlen}\usebox{\alignbox}}}%
%          {\usebox{\alignbox}}}%
%       \vspace*{\alignwidth}\pagebreak[0]\vspace{-\alignwidth}\par}
% % a multi-line comment, whose first argument is its width in points.
%  
   
%%%%%%%%%%%%%%%%%%%%%%%%%%%%%%%%%%%%%%%%%%%%%%%%%%%%%%%%%%%%%%%%%%%%%%%%%%%%%%
% THE lcom ENVIRONMENT                                                       %
% ^^^^^^^^^^^^^^^^^^^^                                                       %
% A multi-line comment with no text to its left is typeset in an lcom        % 
% environment, whose argument is a number representing the indentation       % 
% of the left margin, in points.  All the text of the comment should be      % 
% inside cpar environments.                                                  % 
%%%%%%%%%%%%%%%%%%%%%%%%%%%%%%%%%%%%%%%%%%%%%%%%%%%%%%%%%%%%%%%%%%%%%%%%%%%%%%
\newenvironment{lcom}[1]{%
  \setlength{\lcomindent}{#1pt} % Added for PlusCal handling.
  \par\vspace{.2em}%
  \sloppypar
  \setcounter{pardepth}{0}%
  \footnotesize
  \begin{list}{}{%
    \setlength{\leftmargin}{#1pt}
    \setlength{\labelwidth}{0pt}%
    \setlength{\labelsep}{0pt}%
    \setlength{\itemindent}{0pt}%
    \setlength{\topsep}{0pt}%
    \setlength{\parsep}{0pt}%
    \setlength{\partopsep}{0pt}%
    \setlength{\parskip}{0pt}}
    \item[]}%
  {\end{list}\vspace{.3em}\setlength{\lcomindent}{0pt}%
 }


%%%%%%%%%%%%%%%%%%%%%%%%%%%%%%%%%%%%%%%%%%%%%%%%%%%%%%%%%%%%%%%%%%%%%%%%%%%%%
% THE mcom ENVIRONMENT AND \mutivspace COMMAND                              %
% ^^^^^^^^^^^^^^^^^^^^^^^^^^^^^^^^^^^^^^^^^^^^                              %
%                                                                           %
% A part of the spec containing a right-comment of the form                 %
%                                                                           %
%      xxxx (*************)                                                 %
%      yyyy (* ccccccccc *)                                                 %
%      ...  (* ccccccccc *)                                                 %
%           (* ccccccccc *)                                                 %
%           (* ccccccccc *)                                                 %
%           (*************)                                                 %
%                                                                           %
% is typeset by                                                             %
%                                                                           %
%     XXXX \begin{mcom}{d}                                                  %
%            CCCC ... CCC                                                   %
%          \end{mcom}                                                       %
%     YYYY ...                                                              %
%     \multivspace{n}                                                       %
%                                                                           %
% where the number d is the width in points of the comment, n is the        %
% number of xxxx, yyyy, ...  lines to the left of the comment.              %
% All the text of the comment should be typeset in cpar environments.       %
%                                                                           %
% This puts the comment into a single box (so no page breaks can occur      %
% within it).  The entire box is shaded iff the shading flag is true.       %
%%%%%%%%%%%%%%%%%%%%%%%%%%%%%%%%%%%%%%%%%%%%%%%%%%%%%%%%%%%%%%%%%%%%%%%%%%%%%
\newlength{\xmcomlen}%
\newenvironment{mcom}[1]{%
  \setcounter{pardepth}{0}%
  \hspace{.65em}%
  \begin{lrbox}{\alignbox}\sloppypar%
      \setboolean{shading}{false}%
      \setlength{\boxwidth}{#1pt}%
      \addtolength{\boxwidth}{-.65em}%
      \begin{minipage}[t]{\boxwidth}\footnotesize
      \parskip=0pt\relax}%
       {\end{minipage}\end{lrbox}%
       \setlength{\xmcomlen}{\textwidth}%       % For PlusCal shading
       \addtolength{\xmcomlen}{-\wd\alignbox}%  % For PlusCal shading
       \settodepth{\alignwidth}{\usebox{\alignbox}}%
       \global\setlength{\multicommentdepth}{\alignwidth}%
       \setlength{\boxwidth}{\alignwidth}%      % For PlusCal shading
       \global\addtolength{\alignwidth}{-\maxdepth}%
       \addtolength{\boxwidth}{.1em}%           % For PlusCal shading
      \raisebox{0pt}[0pt][0pt]{%
        \ifthenelse{\boolean{shading}}%
          {\ifpcalshading
             \hspace*{-\xmcomlen}%
             \shadebox{\rule[-\boxwidth]{0pt}{0pt}\hspace*{\xmcomlen}%
                          \usebox{\alignbox}}%
           \else
             \shadebox{\usebox{\alignbox}}
           \fi
          }%
          {\usebox{\alignbox}}}%
       \vspace*{\alignwidth}\pagebreak[0]\vspace{-\alignwidth}\par}
 % a multi-line comment, whose first argument is its width in points.


% \multispace{n} produces the vertical space indicated by "|"s in 
% this situation
%   
%     xxxx (*************)
%     xxxx (* ccccccccc *)
%      |   (* ccccccccc *)
%      |   (* ccccccccc *)
%      |   (* ccccccccc *)
%      |   (*************)
%
% where n is the number of "xxxx" lines.
\newcommand{\multivspace}[1]{\addtolength{\multicommentdepth}{-#1\baselineskip}%
 \addtolength{\multicommentdepth}{1.2em}%
 \ifthenelse{\lengthtest{\multicommentdepth > 0pt}}%
    {\par\vspace{\multicommentdepth}\par}{}}

%\newenvironment{hpar}[2]{%
%  \begin{list}{}{\setlength{\leftmargin}{#1pt}%
%                 \addtolength{\leftmargin}{#2pt}%
%                 \setlength{\itemindent}{-#2pt}%
%                 \setlength{\topsep}{0pt}%
%                 \setlength{\parsep}{0pt}%
%                 \setlength{\partopsep}{0pt}%
%                 \setlength{\parskip}{0pt}%
%                 \addtolength{\labelsep}{0pt}}%
%  \item[]\footnotesize}{\end{list}}
%    %%%%%%%%%%%%%%%%%%%%%%%%%%%%%%%%%%%%%%%%%%%%%%%%%%%%%%%%%%%%%%%%%%%%%%%%
%    % Typesets a sequence of paragraphs like this:                         %
%    %                                                                      %
%    %      left |<-- d1 --> XXXXXXXXXXXXXXXXXXXXXXXX                       %
%    %    margin |           <- d2 -> XXXXXXXXXXXXXXX                       %
%    %           |                    XXXXXXXXXXXXXXX                       %
%    %           |                                                          %
%    %           |                    XXXXXXXXXXXXXXX                       %
%    %           |                    XXXXXXXXXXXXXXX                       %
%    %                                                                      %
%    % where d1 = #1pt and d2 = #2pt, but with no vspace between            %
%    % paragraphs.                                                          %
%    %%%%%%%%%%%%%%%%%%%%%%%%%%%%%%%%%%%%%%%%%%%%%%%%%%%%%%%%%%%%%%%%%%%%%%%%

%%%%%%%%%%%%%%%%%%%%%%%%%%%%%%%%%%%%%%%%%%%%%%%%%%%%%%%%%%%%%%%%%%%%%%
% Commands for repeated characters that produce dashes.              %
%%%%%%%%%%%%%%%%%%%%%%%%%%%%%%%%%%%%%%%%%%%%%%%%%%%%%%%%%%%%%%%%%%%%%%
% \raisedDash{wd}{ht}{thk} makes a horizontal line wd characters wide, 
% raised a distance ht ex's above the baseline, with a thickness of 
% thk em's.
\newcommand{\raisedDash}[3]{\raisebox{#2ex}{\setlength{\alignwidth}{.5em}%
  \rule{#1\alignwidth}{#3em}}}

% The following commands take a single argument n and produce the
% output for n repeated characters, as follows
%   \cdash:    -
%   \tdash:    ~
%   \ceqdash:  =
%   \usdash:   _
\newcommand{\cdash}[1]{\raisedDash{#1}{.5}{.04}}
\newcommand{\usdash}[1]{\raisedDash{#1}{0}{.04}}
\newcommand{\ceqdash}[1]{\raisedDash{#1}{.5}{.08}}
\newcommand{\tdash}[1]{\raisedDash{#1}{1}{.08}}

\newlength{\spacewidth}
\setlength{\spacewidth}{.2em}
\newcommand{\e}[1]{\hspace{#1\spacewidth}}
%% \e{i} produces space corresponding to i input spaces.


%% Alignment-file Commands

\newlength{\alignboxwidth}
\newlength{\alignwidth}
\newsavebox{\alignbox}

% \al{i}{j}{txt} is used in the alignment file to put "%{i}{j}{wd}"
% in the log file, where wd is the width of the line up to that point,
% and txt is the following text.
\newcommand{\al}[3]{%
  \typeout{\%{#1}{#2}{\the\alignwidth}}%
  \cl{#3}}

%% \cl{txt} continues a specification line in the alignment file
%% with text txt.
\newcommand{\cl}[1]{%
  \savebox{\alignbox}{\mbox{$\mbox{}#1\mbox{}$}}%
  \settowidth{\alignboxwidth}{\usebox{\alignbox}}%
  \addtolength{\alignwidth}{\alignboxwidth}%
  \usebox{\alignbox}}

% \fl{txt} in the alignment file begins a specification line that
% starts with the text txt.
\newcommand{\fl}[1]{%
  \par
  \savebox{\alignbox}{\mbox{$\mbox{}#1\mbox{}$}}%
  \settowidth{\alignwidth}{\usebox{\alignbox}}%
  \usebox{\alignbox}}



  
%%%%%%%%%%%%%%%%%%%%%%%%%%%%%%%%%%%%%%%%%%%%%%%%%%%%%%%%%%%%%%%%%%%%%%%%%%%%%
% Ordinarily, TeX typesets letters in math mode in a special math italic    %
% font.  This makes it typeset "it" to look like the product of the         %
% variables i and t, rather than like the word "it".  The following         %
% commands tell TeX to use an ordinary italic font instead.                 %
%%%%%%%%%%%%%%%%%%%%%%%%%%%%%%%%%%%%%%%%%%%%%%%%%%%%%%%%%%%%%%%%%%%%%%%%%%%%%
\ifx\documentclass\undefined
\else
  \DeclareSymbolFont{tlaitalics}{\encodingdefault}{cmr}{m}{it}
  \let\itfam\symtlaitalics
\fi

\makeatletter
\newcommand{\tlx@c}{\c@tlx@ctr\advance\c@tlx@ctr\@ne}
\newcounter{tlx@ctr}
\c@tlx@ctr=\itfam \multiply\c@tlx@ctr"100\relax \advance\c@tlx@ctr "7061\relax
\mathcode`a=\tlx@c \mathcode`b=\tlx@c \mathcode`c=\tlx@c \mathcode`d=\tlx@c
\mathcode`e=\tlx@c \mathcode`f=\tlx@c \mathcode`g=\tlx@c \mathcode`h=\tlx@c
\mathcode`i=\tlx@c \mathcode`j=\tlx@c \mathcode`k=\tlx@c \mathcode`l=\tlx@c
\mathcode`m=\tlx@c \mathcode`n=\tlx@c \mathcode`o=\tlx@c \mathcode`p=\tlx@c
\mathcode`q=\tlx@c \mathcode`r=\tlx@c \mathcode`s=\tlx@c \mathcode`t=\tlx@c
\mathcode`u=\tlx@c \mathcode`v=\tlx@c \mathcode`w=\tlx@c \mathcode`x=\tlx@c
\mathcode`y=\tlx@c \mathcode`z=\tlx@c
\c@tlx@ctr=\itfam \multiply\c@tlx@ctr"100\relax \advance\c@tlx@ctr "7041\relax
\mathcode`A=\tlx@c \mathcode`B=\tlx@c \mathcode`C=\tlx@c \mathcode`D=\tlx@c
\mathcode`E=\tlx@c \mathcode`F=\tlx@c \mathcode`G=\tlx@c \mathcode`H=\tlx@c
\mathcode`I=\tlx@c \mathcode`J=\tlx@c \mathcode`K=\tlx@c \mathcode`L=\tlx@c
\mathcode`M=\tlx@c \mathcode`N=\tlx@c \mathcode`O=\tlx@c \mathcode`P=\tlx@c
\mathcode`Q=\tlx@c \mathcode`R=\tlx@c \mathcode`S=\tlx@c \mathcode`T=\tlx@c
\mathcode`U=\tlx@c \mathcode`V=\tlx@c \mathcode`W=\tlx@c \mathcode`X=\tlx@c
\mathcode`Y=\tlx@c \mathcode`Z=\tlx@c
\makeatother

%%%%%%%%%%%%%%%%%%%%%%%%%%%%%%%%%%%%%%%%%%%%%%%%%%%%%%%%%%
%                THE describe ENVIRONMENT                %
%%%%%%%%%%%%%%%%%%%%%%%%%%%%%%%%%%%%%%%%%%%%%%%%%%%%%%%%%%
%
%
% It is like the description environment except it takes an argument
% ARG that should be the text of the widest label.  It adjusts the
% indentation so each item with label LABEL produces
%%      LABEL             blah blah blah
%%      <- width of ARG ->blah blah blah
%%                        blah blah blah
\newenvironment{describe}[1]%
   {\begin{list}{}{\settowidth{\labelwidth}{#1}%
            \setlength{\labelsep}{.5em}%
            \setlength{\leftmargin}{\labelwidth}% 
            \addtolength{\leftmargin}{\labelsep}%
            \addtolength{\leftmargin}{\parindent}%
            \def\makelabel##1{\rm ##1\hfill}}%
            \setlength{\topsep}{0pt}}%% 
                % Sets \topsep to 0 to reduce vertical space above
                % and below embedded displayed equations
   {\end{list}}

%   For tlatex.TeX
\usepackage{verbatim}
\makeatletter
\def\tla{\let\%\relax%
         \@bsphack
         \typeout{\%{\the\linewidth}}%
             \let\do\@makeother\dospecials\catcode`\^^M\active
             \let\verbatim@startline\relax
             \let\verbatim@addtoline\@gobble
             \let\verbatim@processline\relax
             \let\verbatim@finish\relax
             \verbatim@}
\let\endtla=\@esphack

\let\pcal=\tla
\let\endpcal=\endtla
\let\ppcal=\tla
\let\endppcal=\endtla

% The tlatex environment is used by TLATeX.TeX to typeset TLA+.
% TLATeX.TLA starts its files by writing a \tlatex command.  This
% command/environment sets \parindent to 0 and defines \% to its
% standard definition because the writing of the log files is messed up
% if \% is defined to be something else.  It also executes
% \@computerule to determine the dimensions for the TLA horizonatl
% bars.
\newenvironment{tlatex}{\@computerule%
                        \setlength{\parindent}{0pt}%
                       \makeatletter\chardef\%=`\%}{}


% The notla environment produces no output.  You can turn a 
% tla environment to a notla environment to prevent tlatex.TeX from
% re-formatting the environment.

\def\notla{\let\%\relax%
         \@bsphack
             \let\do\@makeother\dospecials\catcode`\^^M\active
             \let\verbatim@startline\relax
             \let\verbatim@addtoline\@gobble
             \let\verbatim@processline\relax
             \let\verbatim@finish\relax
             \verbatim@}
\let\endnotla=\@esphack

\let\nopcal=\notla
\let\endnopcal=\endnotla
\let\noppcal=\notla
\let\endnoppcal=\endnotla

%%%%%%%%%%%%%%%%%%%%%%%% end of tlatex.sty file %%%%%%%%%%%%%%%%%%%%%%% 
% last modified on Fri  3 August 2012 at 14:23:49 PST by lamport

\begin{document}
\tlatex
\setboolean{shading}{true}
 \@x{\makebox[0pt][r]{\scriptsize 1\hspace{1em}}}\moduleLeftDash\@xx{
 {\MODULE} CASPaxos}\moduleRightDash\@xx{}%
\begin{lcom}{0}%
\begin{cpar}{0}{T}{F}{5.0}{0}{}%
**************************************************************************
\end{cpar}%
\begin{cpar}{1}{F}{F}{0}{0}{}%
 This is an interesting extension of the Single-decree \ensuremath{Paxos}
 algorithm
 to a compare-and-swap type register. The algorithm is very similar to
 \ensuremath{Paxos}, but before starting the ACCEPT phase, proposers are free
 to
 mutate the value. The result is that the \ensuremath{Paxos} instance turns
 from a
 write-once register into a reusable register with atomic semantics, for
 example a compare-and-swap register.
\end{cpar}%
\begin{cpar}{0}{T}{F}{2.5}{0}{}%
**************************************************************************
\end{cpar}%
\end{lcom}%
 \@x{\makebox[0pt][r]{\scriptsize 10\hspace{1em}} {\EXTENDS} Integers ,\,
 FiniteSets}%
\@pvspace{8.0pt}%
\begin{lcom}{0}%
\begin{cpar}{0}{T}{F}{5.0}{0}{}%
**************************************************************************
\end{cpar}%
\begin{cpar}{1}{F}{F}{0}{0}{}%
The data one has to define for the model. In this case, a set of
 possible \ensuremath{Values}, a set of acceptors, and a mutator which maps a
 ballot
 number and a value to a new value.
\end{cpar}%
\vshade{5.0}%
\begin{cpar}{0}{F}{F}{0}{0}{}%
 The \ensuremath{Mutator} is a good approximation of how
 compare-and-swap\mbox{'}ping
 proposers would choose the new values, abstracting away nondeterminism
 by using the ballot number to decide on the new value.
\end{cpar}%
\vshade{5.0}%
\begin{cpar}{0}{F}{F}{0}{0}{}%
For model checking, infinite sets must be avoided. For convenience, we
 more or less explicitly assume that values can be compared.
\end{cpar}%
\begin{cpar}{0}{T}{F}{2.5}{0}{}%
**************************************************************************
\end{cpar}%
\end{lcom}%
 \@x{\makebox[0pt][r]{\scriptsize 24\hspace{1em}} {\CONSTANT} Values ,\,
 Acceptors ,\, Mutator ( \_ ,\, \_ )}%
\@pvspace{8.0pt}%
\begin{lcom}{0}%
\begin{cpar}{0}{T}{F}{5.0}{0}{}%
**************************************************************************
\end{cpar}%
\begin{cpar}{1}{F}{F}{0}{0}{}%
The set of quorums. We automatically construct this by taking all the
 subsets of \ensuremath{Acceptors} which are majorities, \ensuremath{i.e}.
 larger in number than
 the unchosen ones. This is for convenience; any definition for which
 \ensuremath{QuorumAssumption} below holds is valid.
\end{cpar}%
\begin{cpar}{0}{T}{F}{2.5}{0}{}%
**************************************************************************
\end{cpar}%
\end{lcom}%
 \@x{\makebox[0pt][r]{\scriptsize 32\hspace{1em}} Quorums \.{\defeq} \{ S
 \.{\in} {\SUBSET} ( Acceptors ) \.{:} Cardinality ( S ) \.{>} Cardinality (
 Acceptors \.{\,\backslash\,} S ) \}}%
\@pvspace{8.0pt}%
\begin{lcom}{0}%
\begin{cpar}{0}{T}{F}{5.0}{0}{}%
**************************************************************************
\end{cpar}%
\begin{cpar}{1}{F}{F}{0}{0}{}%
The register in this algorithm can repeatedly change its value. For
 simplicity, we don\mbox{'}t let it start from ``no value'' but explicitly
 specify a first value here. Choosing the smallest value is reasonable,
 but you can change this however you like.
\end{cpar}%
\begin{cpar}{0}{T}{F}{2.5}{0}{}%
**************************************************************************
\end{cpar}%
\end{lcom}%
 \@x{\makebox[0pt][r]{\scriptsize 40\hspace{1em}} InitialValue \.{\defeq}
 {\CHOOSE} v \.{\in} Values \.{:} \A\, vv \.{\in} Values \.{:} v \.{\leq} vv}%
\@pvspace{8.0pt}%
\begin{lcom}{0}%
\begin{cpar}{0}{T}{F}{5.0}{0}{}%
**************************************************************************
\end{cpar}%
\begin{cpar}{1}{F}{F}{0}{0}{}%
Sanity check that our defined quorums all have nontrivial pair-wise
 intersections. This is pretty clear with the above definition of
 \ensuremath{Quorums}, but note that you could specify any set of quorums
 (even
 minorities) and the algorithm should work the same way, as long as
 \ensuremath{QuorumAssumption} holds.
\end{cpar}%
\begin{cpar}{0}{T}{F}{2.5}{0}{}%
**************************************************************************
\end{cpar}%
\end{lcom}%
 \@x{\makebox[0pt][r]{\scriptsize 49\hspace{1em}} {\ASSUME} QuorumAssumption
 \.{\defeq} \.{\land} \A\, Q \.{\in} Quorums \.{:} Q \.{\subseteq} Acceptors}%
 \@x{\makebox[0pt][r]{\scriptsize 50\hspace{1em}}\@s{145.19} \.{\land} \A\, Q1
 ,\, Q2 \.{\in} Quorums \.{:} Q1 \.{\cap} Q2 \.{\neq} \{ \}}%
\@pvspace{8.0pt}%
\begin{lcom}{0}%
\begin{cpar}{0}{T}{F}{5.0}{0}{}%
**************************************************************************
\end{cpar}%
\begin{cpar}{1}{F}{F}{0}{0}{}%
Ballot numbers are natural numbers, but it\mbox{'}s good to have an alias so
 that you know when you\mbox{'}re talking about ballots.
\end{cpar}%
\begin{cpar}{0}{T}{F}{2.5}{0}{}%
**************************************************************************
\end{cpar}%
\end{lcom}%
 \@x{\makebox[0pt][r]{\scriptsize 56\hspace{1em}} Ballot\@s{4.1} \.{\defeq}
 Nat}%
\begin{lcom}{0}%
\begin{cpar}{0}{T}{F}{5.0}{0}{}%
**************************************************************************
\end{cpar}%
\begin{cpar}{1}{F}{F}{0}{0}{}%
Now that we know what a ballot is, check that the \ensuremath{Mutator} maps
 Ballots and \ensuremath{Values} to \ensuremath{Values}.
\end{cpar}%
\begin{cpar}{0}{T}{F}{2.5}{0}{}%
**************************************************************************
\end{cpar}%
\end{lcom}%
 \@x{\makebox[0pt][r]{\scriptsize 61\hspace{1em}} {\ASSUME} \.{\land} \A\, b
 \.{\in} Ballot ,\, v \.{\in} Values \.{:} Mutator ( b ,\, v ) \.{\in}
 Values}%
\@pvspace{16.0pt}%
\begin{lcom}{0}%
\begin{cpar}{0}{T}{F}{5.0}{0}{}%
**************************************************************************
\end{cpar}%
\begin{cpar}{1}{F}{F}{0}{0}{}%
Define the set of all possible messages. In this specification,
 proposers are implicit. Messages originating from them are created
 ``out of thin air'' and not addressed to a specific acceptor. In
 practice they would be, though note that each acceptor would receive
 the same ``message body'', and omitting the explicit recipient reduces
 the state space. Note also that messages are not explicitly rejected
 but simply not reacted to. In particular, the implicit proposer has no
 notion of which ballot to try next. The spec lets them try arbitrary
 ballots instead.
\end{cpar}%
\vshade{5.0}%
\begin{cpar}{0}{F}{F}{0}{0}{}%
A message is either a prepare request for a ballot, a prepare response,
 an accept request for a ballot with a new value, or an accept response.
\end{cpar}%
\begin{cpar}{0}{T}{F}{2.5}{0}{}%
**************************************************************************
\end{cpar}%
\end{lcom}%
 \@x{\makebox[0pt][r]{\scriptsize 78\hspace{1em}} Message \.{\defeq}\@s{20.5}
 [ type \.{:} \{\@w{prepare{-}req} \} ,\, bal \.{:} Ballot ]}%
 \@x{\makebox[0pt][r]{\scriptsize 79\hspace{1em}}\@s{49.10} \.{\cup}\@s{15.97}
 [}%
 \@x{\makebox[0pt][r]{\scriptsize 80\hspace{1em}}\@s{80.29} type \.{:}
 \{\@w{prepare{-}rsp} \} ,\, acc \.{:} Acceptors ,\,}%
 \@x{\makebox[0pt][r]{\scriptsize 81\hspace{1em}}\@s{80.29} promised \.{:}
 Ballot ,\,}%
\@y{\@s{0}%
 ballot for which promise is given
}%
\@xx{}%
 \@x{\makebox[0pt][r]{\scriptsize 82\hspace{1em}}\@s{80.29} accepted\@s{2.93}
 \.{:} Ballot ,\,}%
\@y{\@s{0}%
 ballot at which \ensuremath{val} was accepted
}%
\@xx{}%
\@x{\makebox[0pt][r]{\scriptsize 83\hspace{1em}}\@s{80.29} val \.{:} Values}%
\@x{\makebox[0pt][r]{\scriptsize 84\hspace{1em}}\@s{76.19} ]}%
 \@x{\makebox[0pt][r]{\scriptsize 85\hspace{1em}}\@s{49.10} \.{\cup}\@s{15.97}
 [ type \.{:} \{\@w{accept{-}req} \} ,\, bal\@s{0.96} \.{:} Ballot ,\, newVal
 \.{:} Values ]}%
 \@x{\makebox[0pt][r]{\scriptsize 86\hspace{1em}}\@s{49.10} \.{\cup}\@s{15.97}
 [ type \.{:} \{\@w{accept{-}rsp} \} ,\, acc \.{:} Acceptors ,\, accepted
 \.{:} Ballot ]}%
\@x{\makebox[0pt][r]{\scriptsize 87\hspace{1em}}}\midbar\@xx{}%
\@pvspace{8.0pt}%
\begin{lcom}{0}%
\begin{cpar}{0}{T}{F}{5.0}{0}{}%
**************************************************************************
\end{cpar}%
\begin{cpar}{1}{F}{F}{0}{0}{}%
 \ensuremath{{\langle}prepared[a],\, accepted[a],\, value[a]{\rangle}} is the
 state of acceptor a: The
 ballot for which a promise has been made (\ensuremath{i.e}. no smaller
 ballot\mbox{'}s
 value will be accepted); the ballot at which the last value has been
 accepted; and the last accepted value.
\end{cpar}%
\begin{cpar}{0}{T}{F}{2.5}{0}{}%
**************************************************************************
\end{cpar}%
\end{lcom}%
\@x{\makebox[0pt][r]{\scriptsize 95\hspace{1em}} {\VARIABLE} prepared ,\,}%
\@x{\makebox[0pt][r]{\scriptsize 96\hspace{1em}}\@s{46.84} accepted ,\,}%
\@x{\makebox[0pt][r]{\scriptsize 97\hspace{1em}}\@s{46.84} value}%
\@pvspace{8.0pt}%
\begin{lcom}{0}%
\begin{cpar}{0}{T}{F}{5.0}{0}{}%
**************************************************************************
\end{cpar}%
\begin{cpar}{1}{F}{F}{0}{0}{}%
The set of all messages sent. In each state transition of the model, a
 message which could solicit a transaction may be reacted to. Note that
 this implicitly models that a message sent may be received multiple
 times, and that everything can arbitrarily reorder.
\end{cpar}%
\begin{cpar}{0}{T}{F}{2.5}{0}{}%
**************************************************************************
\end{cpar}%
\end{lcom}%
\@x{\makebox[0pt][r]{\scriptsize 105\hspace{1em}} {\VARIABLE} msgs}%
\@pvspace{8.0pt}%
\begin{lcom}{0}%
\begin{cpar}{0}{T}{F}{5.0}{0}{}%
**************************************************************************
\end{cpar}%
\begin{cpar}{1}{F}{F}{0}{0}{}%
An invariant which checks that the variables have values which make sense.
\end{cpar}%
\begin{cpar}{0}{T}{F}{2.5}{0}{}%
**************************************************************************
\end{cpar}%
\end{lcom}%
 \@x{\makebox[0pt][r]{\scriptsize 110\hspace{1em}} TypeOK \.{\defeq} \.{\land}
 prepared \.{\in} [ Acceptors \.{\rightarrow} Ballot ]}%
 \@x{\makebox[0pt][r]{\scriptsize 111\hspace{1em}}\@s{56.14} \.{\land}
 accepted\@s{0.51} \.{\in} [ Acceptors \.{\rightarrow} Ballot ]}%
 \@x{\makebox[0pt][r]{\scriptsize 112\hspace{1em}}\@s{56.14} \.{\land} value
 \.{\in}\@s{12.29} [ Acceptors\@s{1.78} \.{\rightarrow} Values ]}%
 \@x{\makebox[0pt][r]{\scriptsize 113\hspace{1em}}\@s{56.14} \.{\land} msgs
 \.{\subseteq} Message}%
\@pvspace{8.0pt}%
\begin{lcom}{0}%
\begin{cpar}{0}{T}{F}{5.0}{0}{}%
**************************************************************************
\end{cpar}%
\begin{cpar}{1}{F}{F}{0}{0}{}%
The initial state of the model. Note that the state here has an
 initial committed value, ie the register doesn\mbox{'}t start ``empty''. This
 is an inconsequential simplification.
\end{cpar}%
\begin{cpar}{0}{T}{F}{2.5}{0}{}%
**************************************************************************
\end{cpar}%
\end{lcom}%
 \@x{\makebox[0pt][r]{\scriptsize 120\hspace{1em}} Init \.{\defeq} \.{\land}
 prepared \.{=} [ a \.{\in} Acceptors \.{\mapsto} 0 ]}%
 \@x{\makebox[0pt][r]{\scriptsize 121\hspace{1em}}\@s{35.70} \.{\land}
 accepted\@s{0.51} \.{=} [ a \.{\in} Acceptors \.{\mapsto} 0 ]}%
 \@x{\makebox[0pt][r]{\scriptsize 122\hspace{1em}}\@s{35.70} \.{\land}
 value\@s{14.08} \.{=} [ a \.{\in} Acceptors \.{\mapsto} InitialValue ]}%
 \@x{\makebox[0pt][r]{\scriptsize 123\hspace{1em}}\@s{35.70} \.{\land} msgs
 \.{=} \{ \}}%
\@pvspace{8.0pt}%
\begin{lcom}{0}%
\begin{cpar}{0}{T}{F}{5.0}{0}{}%
**************************************************************************
\end{cpar}%
\begin{cpar}{1}{F}{F}{0}{0}{}%
Sending a message just means adding it to the set of all messages.
\end{cpar}%
\begin{cpar}{0}{T}{F}{2.5}{0}{}%
**************************************************************************
\end{cpar}%
\end{lcom}%
 \@x{\makebox[0pt][r]{\scriptsize 128\hspace{1em}} Send ( m ) \.{\defeq} msgs
 \.{'} \.{=} msgs \.{\cup} \{ m \}}%
\@pvspace{8.0pt}%
\begin{lcom}{0}%
\begin{cpar}{0}{T}{F}{5.0}{0}{}%
**************************************************************************
\end{cpar}%
\begin{cpar}{1}{F}{F}{0}{0}{}%
A ballot is started by sending a prepare request (with the hope that
 responses will be received from a quorum). We could allow multiple
 prepare requests for a single ballot, but since they are all identical
 and we already model multiple-receipt for all messages, this adds only
 state space complexity. So a ballot will only be prepared once in this
 model.
\end{cpar}%
\begin{cpar}{0}{T}{F}{2.5}{0}{}%
**************************************************************************
\end{cpar}%
\end{lcom}%
 \@x{\makebox[0pt][r]{\scriptsize 138\hspace{1em}} BallotActive ( b )
 \.{\defeq} \E\, m \.{\in} msgs \.{:}}%
 \@x{\makebox[0pt][r]{\scriptsize 139\hspace{1em}}\@s{102.13} \.{\land} m .
 type \.{=}\@w{prepare{-}req}}%
 \@x{\makebox[0pt][r]{\scriptsize 140\hspace{1em}}\@s{102.13} \.{\land} m .
 bal \.{=} b}%
 \@x{\makebox[0pt][r]{\scriptsize 141\hspace{1em}} PrepareReq ( b )
 \.{\defeq}}%
 \@x{\makebox[0pt][r]{\scriptsize 142\hspace{1em}}\@s{16.4} \.{\land} {\lnot}
 BallotActive ( b )}%
 \@x{\makebox[0pt][r]{\scriptsize 143\hspace{1em}}\@s{16.4} \.{\land} Send (
 [}%
 \@x{\makebox[0pt][r]{\scriptsize 144\hspace{1em}}\@s{69.78} type
 \.{\mapsto}\@w{prepare{-}req} ,\,}%
 \@x{\makebox[0pt][r]{\scriptsize 145\hspace{1em}}\@s{69.78} bal\@s{5.34}
 \.{\mapsto} b}%
\@x{\makebox[0pt][r]{\scriptsize 146\hspace{1em}}\@s{49.49} ] )}%
 \@x{\makebox[0pt][r]{\scriptsize 147\hspace{1em}}\@s{16.4} \.{\land}
 {\UNCHANGED} ( {\langle} prepared ,\, accepted ,\, value {\rangle} )}%
\@pvspace{8.0pt}%
\begin{lcom}{0}%
\begin{cpar}{0}{T}{F}{5.0}{0}{}%
**************************************************************************
\end{cpar}%
\begin{cpar}{1}{F}{F}{0}{0}{}%
A prepare response can be sent if by an acceptor if a) a response was
 demanded via a prepare request and \ensuremath{b}) the acceptor has not
 already
 prepared that or any larger ballot. On success, the acceptor remembers
 that it has prepared the new ballot, and sends to response.
\end{cpar}%
\begin{cpar}{0}{T}{F}{2.5}{0}{}%
**************************************************************************
\end{cpar}%
\end{lcom}%
 \@x{\makebox[0pt][r]{\scriptsize 155\hspace{1em}} PrepareRsp ( a )
 \.{\defeq}}%
 \@x{\makebox[0pt][r]{\scriptsize 156\hspace{1em}}\@s{16.4} \.{\land} \E\, m
 \.{\in} msgs \.{:}}%
 \@x{\makebox[0pt][r]{\scriptsize 157\hspace{1em}}\@s{31.61} \.{\land} m .
 type \.{=}\@w{prepare{-}req}}%
 \@x{\makebox[0pt][r]{\scriptsize 158\hspace{1em}}\@s{31.61} \.{\land} m . bal
 \.{>} prepared [ a ]}%
 \@x{\makebox[0pt][r]{\scriptsize 159\hspace{1em}}\@s{31.61} \.{\land}
 prepared \.{'} \.{=} [ prepared {\EXCEPT} {\bang} [ a ] \.{=} m . bal ]}%
 \@x{\makebox[0pt][r]{\scriptsize 160\hspace{1em}}\@s{31.61} \.{\land} Send (
 [}%
 \@x{\makebox[0pt][r]{\scriptsize 161\hspace{1em}}\@s{84.99} acc\@s{25.12}
 \.{\mapsto} a ,\,}%
 \@x{\makebox[0pt][r]{\scriptsize 162\hspace{1em}}\@s{84.99} type\@s{21.36}
 \.{\mapsto}\@w{prepare{-}rsp} ,\,}%
 \@x{\makebox[0pt][r]{\scriptsize 163\hspace{1em}}\@s{84.99} promised
 \.{\mapsto} m . bal ,\,}%
 \@x{\makebox[0pt][r]{\scriptsize 164\hspace{1em}}\@s{84.99} accepted\@s{2.93}
 \.{\mapsto} accepted [ a ] ,\,}%
 \@x{\makebox[0pt][r]{\scriptsize 165\hspace{1em}}\@s{84.99} val\@s{26.19}
 \.{\mapsto} value [ a ]}%
\@x{\makebox[0pt][r]{\scriptsize 166\hspace{1em}}\@s{64.71} ] )}%
 \@x{\makebox[0pt][r]{\scriptsize 167\hspace{1em}}\@s{16.4} \.{\land}
 {\UNCHANGED} {\langle} accepted ,\, value {\rangle}}%
\@pvspace{8.0pt}%
\begin{lcom}{0}%
\begin{cpar}{0}{T}{F}{5.0}{0}{}%
**************************************************************************
\end{cpar}%
\begin{cpar}{1}{F}{F}{0}{0}{}%
 An accept request can only be sent (\ensuremath{i.e}. be fabricated from thin
 air)
 if a) once; \ensuremath{b}) if prepare responses for the ballot have been
 received
 from a quorum; \ensuremath{c}) with a new value based on the most recently
 accepted
 value from the prepare responses.
\end{cpar}%
\begin{cpar}{0}{T}{F}{2.5}{0}{}%
**************************************************************************
\end{cpar}%
\end{lcom}%
 \@x{\makebox[0pt][r]{\scriptsize 175\hspace{1em}} AcceptReq ( b ,\, v )
 \.{\defeq}}%
 \@x{\makebox[0pt][r]{\scriptsize 176\hspace{1em}}\@s{16.4} \.{\land} {\lnot}
 \E\, m \.{\in} msgs \.{:} m . type \.{=}\@w{accept{-}req} \.{\land} m . bal
 \.{=} b}%
 \@x{\makebox[0pt][r]{\scriptsize 177\hspace{1em}}\@s{16.4} \.{\land} \E\, Q
 \.{\in} Quorums \.{:}}%
 \@x{\makebox[0pt][r]{\scriptsize 178\hspace{1em}}\@s{31.61} \.{\LET} M
 \.{\defeq} \{ m \.{\in} msgs \.{:} \.{\land} m . type
 \.{=}\@w{prepare{-}rsp}}%
 \@x{\makebox[0pt][r]{\scriptsize 179\hspace{1em}}\@s{138.89} \.{\land} m .
 promised \.{=} b}%
 \@x{\makebox[0pt][r]{\scriptsize 180\hspace{1em}}\@s{138.89} \.{\land} m .
 acc \.{\in} Q \}}%
 \@x{\makebox[0pt][r]{\scriptsize 181\hspace{1em}}\@s{31.61} \.{\IN} \.{\land}
 \A\, a\@s{3.06} \.{\in} Q\@s{1.99} \.{:} \E\, m \.{\in} M \.{:} m . acc
 \.{=} a}%
 \@x{\makebox[0pt][r]{\scriptsize 182\hspace{1em}}\@s{52.01} \.{\land} \E\, m
 \.{\in} M \.{:}}%
 \@x{\makebox[0pt][r]{\scriptsize 183\hspace{1em}}\@s{71.32} \.{\land} m . val
 \.{=} v}%
 \@x{\makebox[0pt][r]{\scriptsize 184\hspace{1em}}\@s{71.32} \.{\land} \A\,
 mm\@s{0.67} \.{\in} M \.{:} mm . accepted \.{\leq} m . accepted}%
 \@x{\makebox[0pt][r]{\scriptsize 185\hspace{1em}}\@s{16.4} \.{\land} \.{\LET}
 newVal \.{\defeq} Mutator ( b ,\, v )}%
\@y{\@s{0}%
 crucial difference from \ensuremath{Paxos
}}%
\@xx{}%
\@x{\makebox[0pt][r]{\scriptsize 186\hspace{1em}}\@s{27.51} \.{\IN} Send ( [}%
 \@x{\makebox[0pt][r]{\scriptsize 187\hspace{1em}}\@s{77.88} type\@s{14.10}
 \.{\mapsto}\@w{accept{-}req} ,\,}%
 \@x{\makebox[0pt][r]{\scriptsize 188\hspace{1em}}\@s{77.88} bal\@s{19.44}
 \.{\mapsto} b ,\,}%
 \@x{\makebox[0pt][r]{\scriptsize 189\hspace{1em}}\@s{77.88} newVal
 \.{\mapsto} newVal}%
\@x{\makebox[0pt][r]{\scriptsize 190\hspace{1em}}\@s{73.78} ] )}%
 \@x{\makebox[0pt][r]{\scriptsize 191\hspace{1em}}\@s{16.4} \.{\land}
 {\UNCHANGED} ( {\langle} accepted ,\, value ,\, prepared {\rangle} )}%
\@pvspace{8.0pt}%
\begin{lcom}{0}%
\begin{cpar}{0}{T}{F}{5.0}{0}{}%
**************************************************************************
\end{cpar}%
\begin{cpar}{1}{F}{F}{0}{0}{}%
An acceptor can reply to an accept request only if it hasn\mbox{'}t yet
 prepared a higher ballot. Before replying, it makes sure it marks the
 ballot as prepared (as the particular acceptor may not have received
 the associated prepare request earlier), and updates its accepted
 ballot and the new value.
\end{cpar}%
\begin{cpar}{0}{T}{F}{2.5}{0}{}%
**************************************************************************
\end{cpar}%
\end{lcom}%
\@x{\makebox[0pt][r]{\scriptsize 200\hspace{1em}} AcceptRsp ( a ) \.{\defeq}}%
 \@x{\makebox[0pt][r]{\scriptsize 201\hspace{1em}}\@s{16.4} \.{\land} \E\, m
 \.{\in} msgs \.{:}}%
 \@x{\makebox[0pt][r]{\scriptsize 202\hspace{1em}}\@s{31.61} \.{\land} m .
 type \.{=}\@w{accept{-}req}}%
 \@x{\makebox[0pt][r]{\scriptsize 203\hspace{1em}}\@s{31.61} \.{\land} m . bal
 \.{\geq} prepared [ a ]}%
 \@x{\makebox[0pt][r]{\scriptsize 204\hspace{1em}}\@s{31.61} \.{\land}
 prepared \.{'} \.{=} [ prepared {\EXCEPT} {\bang} [ a ] \.{=} m . bal ]}%
 \@x{\makebox[0pt][r]{\scriptsize 205\hspace{1em}}\@s{31.61} \.{\land}
 accepted \.{'}\@s{0.51} \.{=} [ accepted {\EXCEPT} {\bang} [ a ]\@s{0.51}
 \.{=} m . bal ]}%
 \@x{\makebox[0pt][r]{\scriptsize 206\hspace{1em}}\@s{31.61} \.{\land} value
 \.{'}\@s{14.08} \.{=} [ value\@s{12.29} {\EXCEPT} {\bang} [ a ]\@s{1.78}
 \.{=} m . newVal ]}%
 \@x{\makebox[0pt][r]{\scriptsize 207\hspace{1em}}\@s{31.61} \.{\land} Send (
 [}%
 \@x{\makebox[0pt][r]{\scriptsize 208\hspace{1em}}\@s{84.99} acc\@s{22.19}
 \.{\mapsto} a ,\,}%
 \@x{\makebox[0pt][r]{\scriptsize 209\hspace{1em}}\@s{84.99} type\@s{18.42}
 \.{\mapsto}\@w{accept{-}rsp} ,\,}%
 \@x{\makebox[0pt][r]{\scriptsize 210\hspace{1em}}\@s{84.99} accepted
 \.{\mapsto} m . bal}%
\@x{\makebox[0pt][r]{\scriptsize 211\hspace{1em}}\@s{68.59} ] )}%
\@pvspace{8.0pt}%
\begin{lcom}{0}%
\begin{cpar}{0}{T}{F}{5.0}{0}{}%
**************************************************************************
\end{cpar}%
\begin{cpar}{1}{F}{F}{0}{0}{}%
Next is true if and only if the new state (\ensuremath{i.e}. that with primed
 variables) is valid. This is used to simulate the model by
 constructing new states until we run out of options. Concretely, the
 below means that either we prepare a ballot, or can react successfully
 to prepare request, or there is an acceptor which can find a message it
 can react to.
\end{cpar}%
\begin{cpar}{0}{T}{F}{2.5}{0}{}%
**************************************************************************
\end{cpar}%
\end{lcom}%
 \@x{\makebox[0pt][r]{\scriptsize 221\hspace{1em}} Next \.{\defeq} \.{\lor}
 \E\, b\@s{0.64} \.{\in} Ballot \.{:} \.{\lor} PrepareReq ( b )}%
 \@x{\makebox[0pt][r]{\scriptsize 222\hspace{1em}}\@s{112.86} \.{\lor} \E\, v
 \.{\in} Values \.{:} AcceptReq ( b ,\, v )}%
 \@x{\makebox[0pt][r]{\scriptsize 223\hspace{1em}}\@s{39.83} \.{\lor} \E\, a
 \.{\in} Acceptors \.{:} PrepareRsp ( a ) \.{\lor} AcceptRsp ( a )}%
\@pvspace{8.0pt}%
\begin{lcom}{0}%
\begin{cpar}{0}{T}{F}{5.0}{0}{}%
**************************************************************************
\end{cpar}%
\begin{cpar}{1}{F}{F}{0}{0}{}%
Spec is the (default) entry point for the TLA+ model runner. The below
 formula is a temporal formula and means that the valid behaviors of the
 specification are those which initially satisfy \ensuremath{Init}, and from
 each
 step to the following the formula \ensuremath{Next} is satisfied, unless
 none of the
 variables changes (which is called a ``stuttering step''). The model
 runner uses this to expand all possible behaviors.
\end{cpar}%
\begin{cpar}{0}{T}{F}{2.5}{0}{}%
**************************************************************************
\end{cpar}%
\end{lcom}%
 \@x{\makebox[0pt][r]{\scriptsize 233\hspace{1em}} Spec \.{\defeq} Init
 \.{\land} {\Box} [ Next ]_{ {\langle} prepared ,\, accepted ,\, value ,\,
 msgs {\rangle}}}%
\@pvspace{8.0pt}%
\begin{lcom}{0}%
\begin{cpar}{0}{T}{F}{5.0}{0}{}%
**************************************************************************
\end{cpar}%
\begin{cpar}{1}{F}{F}{0}{0}{}%
Equipped with a model, what invariants do we want to hold\.{?} Or, in other
 words, what exactly is it that we think the algorithm guarantees\.{?}
 Naively,
 each newly committed value should be in some relation to a previous value,
 so when you\mbox{'}re not thinking to hard about it, you could hope that
 when you
 take a committed value A and the previously committed value \ensuremath{B},
 then \ensuremath{B} was
 created by mutating A. It\mbox{'}s not quite as simple, but going down that
 wrong
 track highlights how to use the model checker to find interesting histories.
\end{cpar}%
\vshade{5.0}%
\begin{cpar}{0}{F}{F}{0}{0}{}%
 If you have a minute, figure out why the above assumption is wrong. You
 don\mbox{'}t
 need more than three ballots and two concurrent proposals.
\end{cpar}%
\begin{cpar}{0}{T}{F}{2.5}{0}{}%
**************************************************************************
\end{cpar}%
\vshade{5.0}%
\begin{cpar}{1}{F}{F}{0}{0}{}%
**************************************************************************
\end{cpar}%
\begin{cpar}{0}{F}{F}{0}{0}{}%
For a given ballot, find the acceptors which (at some point in time)
 accepted that ballot.
\end{cpar}%
\begin{cpar}{0}{T}{F}{2.5}{0}{}%
**************************************************************************
\end{cpar}%
\end{lcom}%
 \@x{\makebox[0pt][r]{\scriptsize 252\hspace{1em}} AcceptedBy ( b ) \.{\defeq}
 \{ a \.{\in} Acceptors \.{:}}%
 \@x{\makebox[0pt][r]{\scriptsize 253\hspace{1em}}\@s{16.4} \E\, m \.{\in}
 msgs \.{:} \.{\land} m . type \.{=}\@w{accept{-}rsp}}%
 \@x{\makebox[0pt][r]{\scriptsize 254\hspace{1em}}\@s{76.00} \.{\land} m . acc
 \.{=} a}%
 \@x{\makebox[0pt][r]{\scriptsize 255\hspace{1em}}\@s{76.00} \.{\land} m .
 accepted \.{=} b \}}%
\@pvspace{8.0pt}%
\begin{lcom}{0}%
\begin{cpar}{0}{T}{F}{5.0}{0}{}%
**************************************************************************
\end{cpar}%
\begin{cpar}{1}{F}{F}{0}{0}{}%
For a given ballot, find out whether the ballot was ever accepted by a
 quorum.
\end{cpar}%
\begin{cpar}{0}{T}{F}{2.5}{0}{}%
**************************************************************************
\end{cpar}%
\end{lcom}%
 \@x{\makebox[0pt][r]{\scriptsize 261\hspace{1em}} AcceptedByQuorum ( b )
 \.{\defeq} \E\, Q \.{\in} Quorums \.{:} AcceptedBy ( b ) \.{\cap} Q \.{=} Q}%
\@pvspace{8.0pt}%
\begin{lcom}{0}%
\begin{cpar}{0}{T}{F}{5.0}{0}{}%
**************************************************************************
\end{cpar}%
\begin{cpar}{1}{F}{F}{0}{0}{}%
The set of committed ballot numbers. Note that 0 is trivially
 committed thanks to the initialization of the state.
\end{cpar}%
\begin{cpar}{0}{T}{F}{2.5}{0}{}%
**************************************************************************
\end{cpar}%
\end{lcom}%
 \@x{\makebox[0pt][r]{\scriptsize 267\hspace{1em}} CommittedBallots \.{\defeq}
 \{ b \.{\in} Ballot \.{:} AcceptedByQuorum ( b ) \} \.{\cup} \{ 0 \}}%
\@pvspace{8.0pt}%
\begin{lcom}{0}%
\begin{cpar}{0}{T}{F}{5.0}{0}{}%
**************************************************************************
\end{cpar}%
\begin{cpar}{1}{F}{F}{0}{0}{}%
 For a given ballot \ensuremath{b \.{>} 0}, find the next lowest ballot number
 which was
 committed (note that this doesn\mbox{'}t have to be \ensuremath{b}-1).
\end{cpar}%
\begin{cpar}{0}{T}{F}{2.5}{0}{}%
**************************************************************************
\end{cpar}%
\end{lcom}%
 \@x{\makebox[0pt][r]{\scriptsize 273\hspace{1em}} BallotCommittedBefore ( b )
 \.{\defeq} {\CHOOSE} c \.{\in} CommittedBallots \.{:}}%
 \@x{\makebox[0pt][r]{\scriptsize 274\hspace{1em}}\@s{150.90}
 \.{\land}\@s{11.17} c \.{<} b}%
 \@x{\makebox[0pt][r]{\scriptsize 275\hspace{1em}}\@s{150.90}
 \.{\land}\@s{11.17} \A\, cc \.{\in} CommittedBallots \.{:}}%
 \@x{\makebox[0pt][r]{\scriptsize 276\hspace{1em}}\@s{177.29} cc \.{\geq} b
 \.{\lor} cc \.{\leq} c}%
\@pvspace{8.0pt}%
\begin{lcom}{0}%
\begin{cpar}{0}{T}{F}{5.0}{0}{}%
**************************************************************************
\end{cpar}%
\begin{cpar}{1}{F}{F}{0}{0}{}%
For a given ballot, collect all the values which an acceptor requested.
 This set will always either be empty or a singleton, but that\mbox{'}s not
 something you can tell from this definition, though we assert it below.
\end{cpar}%
\begin{cpar}{0}{T}{F}{2.5}{0}{}%
**************************************************************************
\end{cpar}%
\end{lcom}%
 \@x{\makebox[0pt][r]{\scriptsize 283\hspace{1em}} ValuesAt ( b ) \.{\defeq}
 {\IF} b \.{=} 0 \.{\THEN} \{ InitialValue \}}%
 \@x{\makebox[0pt][r]{\scriptsize 284\hspace{1em}}\@s{71.73} \.{\ELSE} \{ v
 \.{\in} Values \.{:}}%
 \@x{\makebox[0pt][r]{\scriptsize 285\hspace{1em}}\@s{113.72} \E\, m \.{\in}
 msgs \.{:}}%
 \@x{\makebox[0pt][r]{\scriptsize 286\hspace{1em}}\@s{125.04} \.{\land} m .
 type\@s{18.42} \.{=}\@w{accept{-}rsp}}%
 \@x{\makebox[0pt][r]{\scriptsize 287\hspace{1em}}\@s{125.04} \.{\land} m .
 accepted \.{=} b}%
 \@x{\makebox[0pt][r]{\scriptsize 288\hspace{1em}}\@s{125.04} \.{\land} \E\,
 mm \.{\in} msgs \.{:}}%
 \@x{\makebox[0pt][r]{\scriptsize 289\hspace{1em}}\@s{140.25} \.{\land} mm .
 type\@s{14.10} \.{=}\@w{accept{-}req}}%
 \@x{\makebox[0pt][r]{\scriptsize 290\hspace{1em}}\@s{140.25} \.{\land} mm .
 bal\@s{19.44} \.{=} b}%
 \@x{\makebox[0pt][r]{\scriptsize 291\hspace{1em}}\@s{140.25} \.{\land} mm .
 newVal \.{=} v}%
\@x{\makebox[0pt][r]{\scriptsize 292\hspace{1em}}\@s{103.04} \}}%
 \@x{\makebox[0pt][r]{\scriptsize 293\hspace{1em}} OnlyOneValuePerBallot
 \.{\defeq} \A\, b \.{\in} Ballot \.{:} Cardinality ( ValuesAt ( b ) )
 \.{\leq} 1}%
\@pvspace{8.0pt}%
\begin{lcom}{0}%
\begin{cpar}{0}{T}{F}{5.0}{0}{}%
**************************************************************************
\end{cpar}%
\begin{cpar}{1}{F}{F}{0}{0}{}%
 \ensuremath{MutationsLineUp} is the main (ill-fated) assertion that each new
 value
 of the register is based on a previously committed value.
\end{cpar}%
\begin{cpar}{0}{T}{F}{2.5}{0}{}%
**************************************************************************
\end{cpar}%
\end{lcom}%
 \@x{\makebox[0pt][r]{\scriptsize 299\hspace{1em}} UnwrapSingleton ( s )
 \.{\defeq} {\CHOOSE} v \.{\in} s \.{:} {\TRUE}}%
\@y{\@s{0}%
 \ensuremath{\{x\} \.{\mapsto} x
}}%
\@xx{}%
\@x{\makebox[0pt][r]{\scriptsize 300\hspace{1em}} MutationsLineUp \.{\defeq}}%
 \@x{\makebox[0pt][r]{\scriptsize 301\hspace{1em}}\@s{16.4} \A\, b \.{\in}
 CommittedBallots \.{\,\backslash\,} \{ 0 \} \.{:}}%
 \@x{\makebox[0pt][r]{\scriptsize 302\hspace{1em}}\@s{27.72} \.{\LET} newVal
 \.{\defeq} UnwrapSingleton ( ValuesAt ( b ) )}%
 \@x{\makebox[0pt][r]{\scriptsize 303\hspace{1em}}\@s{48.12} prevCommitBallot
 \.{\defeq} BallotCommittedBefore ( b )}%
 \@x{\makebox[0pt][r]{\scriptsize 304\hspace{1em}}\@s{52.22} oldVal \.{\defeq}
 UnwrapSingleton ( ValuesAt ( prevCommitBallot ) )}%
 \@x{\makebox[0pt][r]{\scriptsize 305\hspace{1em}}\@s{27.72} \.{\IN}\@s{4.09}
 newVal \.{=} Mutator ( b ,\, oldVal )}%
\@pvspace{8.0pt}%
\begin{lcom}{0}%
\begin{cpar}{0}{T}{F}{5.0}{0}{}%
**************************************************************************
\end{cpar}%
\begin{cpar}{1}{F}{F}{0}{0}{}%
 \ensuremath{DesiredProperties} is a formula that we will tell the model
 checker to
 verify for each state in all valid behaviors. When it is something
 that actually holds, folks usually call it \ensuremath{Inv} (meaning an
 inductive
 invariant), and may try to prove it mechanically using the TLA proof
 checker, but it is a lot of work and often nontrivial.
\end{cpar}%
\vshade{5.0}%
\begin{cpar}{0}{F}{F}{0}{0}{}%
 In our case, there will be behaviors that violate
 \ensuremath{MutationsLineUp}.
\end{cpar}%
\begin{cpar}{0}{T}{F}{2.5}{0}{}%
**************************************************************************
\end{cpar}%
\end{lcom}%
 \@x{\makebox[0pt][r]{\scriptsize 316\hspace{1em}} DesiredProperties
 \.{\defeq} \.{\land} TypeOK}%
 \@x{\makebox[0pt][r]{\scriptsize 317\hspace{1em}}\@s{96.01} \.{\land}
 OnlyOneValuePerBallot}%
 \@x{\makebox[0pt][r]{\scriptsize 318\hspace{1em}}\@s{96.01} \.{\land}
 MutationsLineUp}%
\@x{\makebox[0pt][r]{\scriptsize 319\hspace{1em}}}\bottombar\@xx{}%
\setboolean{shading}{false}
\begin{lcom}{0}%
\begin{cpar}{0}{F}{F}{0}{0}{}%
\ensuremath{\.{\,\backslash\,}}* Modification History
\end{cpar}%
\begin{cpar}{0}{F}{F}{0}{0}{}%
 \ensuremath{\.{\,\backslash\,}}* Last modified \ensuremath{Fri}
 \ensuremath{Apr} 07 02:26:16 \ensuremath{EDT} 2017 by \ensuremath{tschottdorf
}%
\end{cpar}%
\begin{cpar}{0}{F}{F}{0}{0}{}%
 \ensuremath{\.{\,\backslash\,}}* Created \ensuremath{Thu} \ensuremath{Apr} 06
 02:12:06 \ensuremath{EDT} 2017 by \ensuremath{tschottdorf
}%
\end{cpar}%
\end{lcom}%
\end{document}
